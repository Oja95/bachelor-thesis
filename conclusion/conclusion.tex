\documentclass[..thesis.tex]{subfiles}

\begin{document}

The thesis thorougly explains the ideas behind sampling profiling and describes the problems that could affect such profiling method. It brings out the importance of sufficient samples' quantity and describes safepoint bias, periodicity bias and observer effect. The thesis then dissects the internals of Honest Profiler, describes its benefits over other profiling solutions and explains why it is a viable choice as a profiler when compared to other existing profiling solutions. The performance of Honest Profiler's information acquiring logic is measured. The performance measurements did not reveal any fundamental reasons on why acquiring a greater amount of samples is not possible. The thesis then investigates alternative methods to increase the amount of information that Honest Profiler can extract from the observed application.

The main results of this thesis are three solutions which increase the amount of profiling information extracted without causing significant performance overhead. The solutions' performance and extracted information amount was tested on a benchmark test to verify the results. The first approach used real time based time measurement instead of CPU time based time measurement. The second solution focused on increasing the system's interrupt frequency which, in consequence, increased the amount of samples extracted. As a part of the last solution, an external timekeeping utility was implemented which served as a high frequency CPU time based timer for the Honest Profiler. All of the solutions successfully managed to increase the amount of useful samples extracted from the profileable application.

The solution with increased kernel's frequency and the solution with external timekeeping utility proved to produce reliable results by increasing the amount of profiling information extracted with negligible performance overhead. The real time based solution also increased the amount of useful information extracted but introduced a larger, albeit acceptable performance overhead.
Two of the provided solutions have a different distribution of useful and non-useful samples when compared to the original implementation. Ideas for investigating the distribution differences are presented in the future research Section \ref{sec:future-research}
%\TODO{maini kõiki kolme enda lahendust ja iga juures maini ära, et sai a la oluliselt rohkem useful sampleid kui originaalne, aga suurem osa kõikidest samplidest olid nonuseful vms}



\end{document}
