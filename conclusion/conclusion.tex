\documentclass[..thesis.tex]{subfiles}

\begin{document}

The thesis thorougly explains the ideas behind sampling profiling and describes the problems that could affect such profiling method. It then dissects the internals of Honest Profiler, describes its benefits over other profiling solutions and explains why it is a viable choice as a profiler when compared to other existing profiling solutions. The performance of Honest Profiler's information acquiring logic is measured. The performance measurements did not reveal any fundamental reasons on why acquiring a greater amount of samples is not possible. The thesis then investigates alternative methods to increase the amount of information that  Honest Profiler can extract from the observed application.
\TODO{maini neid osasid, milles pikalt töös keskendusid, eraldi ka kokkuvõttes. maini kõík probleemid eraldi ära ja et a la selgitasid neid ja tõid näiteid}

The main result of this thesis are three solutions which increase the profiling information output wihtout causing noticeable performance overhead. The solution's performance and extracted information amount was tested on a benchmark test to verify the results. The provided solutions proved to produce reliable results with increased information output and negligible performance overhead. Two of the provided solutions have a different distribution of useful and non-useful samples when compared to the original implementation. Ideas for investigating the distribution differences are presented in the future research Section \ref{sec:future-research}
\TODO{maini kõiki kolme enda lahendust ja iga juures maini ära, et sai a la oluliselt rohkem useful sampleid kui originaalne, aga suurem osa kõikidest samplidest olid nonuseful vms}

\end{document}
