\documentclass{style/thesis}
\newcommand{\articleName}{Optimizing JVM profiling performance for Honest Profiler}
\newcommand{\articleNameEE}{JVM profileerimise jõudluse optimeerimine Honest Profileri näitel}

% Nice TODO red box
\newcommand{\TODO}{\todo[inline]}
\newcommand{\supervisor}[2][1=]{\todo[inline,linecolor=puprle, backgroundcolor=purple!25, bordercolor=purple, #1]{#2}}

\graphicspath{ {images/} }

% BEGIN DOCUMENT
\begin{document}

% BEGIN TITLE PAGE
\thispagestyle{empty}
\begin{center}

\large
UNIVERSITY OF TARTU\\[2mm]
Institute of Computer Science\\
Computer Science Curriculum\\[2mm]

\vspace{25mm}

\Large Tiit Oja

\vspace{4mm}

\huge \articleName

\vspace{20mm}

\Large Bachelor's Thesis (9 ECTS)

\end{center}

\vspace{2mm}

\begin{flushright}
 {
 \setlength{\extrarowheight}{5pt}
 \begin{tabular}{r l} 
  \sffamily Supervisor: & \sffamily Vootele Rõtov, MSc \\
  \sffamily Supervisor: & \sffamily Vesal Vojdani, PhD
 \end{tabular}
 }
\end{flushright}

\vfill
\centerline{Tartu 2018}

% END TITLE PAGE

% Remember to remove this from the final thesis version
%\pagebreak
%\listoftodos[There's stuff to do!]
% END OF TODO PAGE 

% COMPULSORY INFO PAGE
\pagebreak

\selectlanguage{english}
\noindent\textbf{\large \articleName}
\vspace*{0mm}
%\begin{flushleft}

\textbf{Abstract:} Honest Profiler is a profiling tool which extracts performance information from applications running on the Java Virtual Machine. This information helps to locate the performance bottlenecks in the application observed. This thesis aims to provide solutions to increase the output of useful information by Honest Profiler. Achieving this would increase the efficiency and accuracy of the profiling results collected by Honest Profiler. Thesis will cover the basics of sampling profiling, the architecture of Honest Profiler and measures the performance of Honest Profiler's data collection logic. As a main result of this thesis, three different solutions for increasing the profiler information output are presented and their performance and the extracted information amount is evaluated by a benchmark test.
  

\vspace*{3ex}
%\begin{flushleft}
  \textbf{Keywords:} 
  Profiling, optimization, Honest Profiler
%\end{flushleft}
\vspace*{3ex}

\noindent\textbf{CERCS:} 
  P170, Computer science, numerical analysis, systems, control 
\selectlanguage{estonian}

\vspace*{5ex}
\noindent\textbf{\large \articleNameEE}
\vspace*{0mm}

\noindent\textbf{Lühikokkuvõte:} Honest Profiler on tööriist, mis võimaldab mõõta Java virtuaalmasina peal jooksvate rakenduste jõudlust. Tööriista poolt kogutud informatsioon annab ülevaate rakenduse jõudlusest, mille baasil on võimalik optimeerida vaadeldava rakenduse jõudlust. Käesoleva töö eesmärk on luua lahendusi, mis suurendaksid Honest Profileri tööriista poolt kogutud informatsiooni hulka. Suurem kogutud informatsiooni hulk muudab jõudluse mõõtmise efektiivsemaks ning saadud tulemused täpsemaks. Töö kirjeldab profiilide kogumise ning Honest Profileri arhitektuuri põhitõdesid ning mõõdab Honest Profileri informatsiooni kogumise loogika jõudlust. Töö põhitulem on kolm erinevat lähenemist, mis suurendavad kogutud informatsiooni hulka. Kirjeldatud lahenduste jõudlus ning täpsus verifitseeritakse jõudlustesti abil.
\vspace*{3ex}

\begin{flushleft}
  \textbf{Võtmesõnad:} 
  Jõudluse mõõtmine, optimeerimine, Honest Profiler
\end{flushleft}
\vspace*{3ex}

\noindent\textbf{CERCS:} 
  P170, Arvutiteadus, arvutusmeetodid, süsteemid, juhtimine (automaatjuhtimisteooria)
\newpage

\selectlanguage{english}
\tableofcontents

\newpage
%Introduction
\section{Introduction}
\subfile{introduction/introduction.tex}

\pagebreak

\section{Sampling profiling methodologies}
\label{sec:profiling}
\subfile{profiling/profiling.tex}

\pagebreak


\section{Honest Profiler}
\label{sec:honest-profiler}
\subfile{honest-profiler/honest-profiler.tex}
\pagebreak

%%\subsection{\texttt{AsyncGetCallTrace} performance analysis}
\section{Honest Profiler performance analysis}
\label{sec:perf-analysis}
\subfile{perf-analysis/perf-analysis.tex}
\pagebreak



\section{Profiler optimization}
\label{sec:optimization}
\subfile{optimization/optimization.tex}

\pagebreak
\section{Future research}
\label{sec:future-research}
\subfile{future-research/future-research.tex}

\clearpage
\section{Conclusion} 

\subfile{conclusion/conclusion.tex}

\newpage

\def\urlprefix{}
\nocite{*}
\bibliography{thesis}{}
\bibliographystyle{plainurl}


\newpage

\appendix
\section*{Appendices}
\addcontentsline{toc}{section}{Appendices}
% So that appendecies would be named by letters!
\renewcommand{\thesubsection}{\Alph{subsection}}
\subfile{appendix/appendix.tex}
\pagebreak
\section*{Non-exclusive licence to reproduce thesis and make thesis public}

% https://tex.stackexchange.com/a/78845
\renewcommand{\labelenumii}{\theenumii}
\renewcommand{\theenumii}{\theenumi.\arabic{enumii}.}

I, \textbf{Tiit Oja},

\begin{enumerate}
	\item herewith grant the University of Tartu a free permit (non-exclusive licence) to:

	\begin{enumerate}[label*=\arabic*.]
		\item reproduce, for the purpose of preservation and making available to the public, including for addition to the DSpace digital archives until expiry of the term of validity of the copyright, and

		\item make available to the public via the web environment of the University of Tartu, including via the DSpace digital archives until expiry of the term of validity of the copyright,
	\end{enumerate}

	\textbf{\articleName},

	supervised by \textbf{Vootele Rõtov} and \textbf{Vesal Vojdani},

	\item I am aware of the fact that the author retains these rights.
	\item I certify that granting the non-exclusive licence does not infringe the intellectual property rights or rights arising from the Personal Data Protection Act.
\end{enumerate}

\noindent
Tartu, \dotdate\today

\end{document}
