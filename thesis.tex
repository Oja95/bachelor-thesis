\documentclass{style/thesis}
\newcommand{\articleName}{Optimizing JVM profiling performance for Honest Profiler}
\newcommand{\articleNameEE}{JVM profileerimise jõudluse optimeerimine Honest Profileri näitel}

% Nice TODO red box
\newcommand{\TODO}{\todo[inline]}
\newcommand{\supervisor}[2][1=]{\todo[inline,linecolor=puprle, backgroundcolor=purple!25, bordercolor=purple, #1]{#2}}

\graphicspath{ {images/} }

% BEGIN DOCUMENT
\begin{document}

% BEGIN TITLE PAGE
\thispagestyle{empty}
\begin{center}

\large
UNIVERSITY OF TARTU\\[2mm]
Institute of Computer Science\\
Computer Science Curriculum\\[2mm]

\vspace{25mm}

\Large Tiit Oja

\vspace{4mm}

\huge \articleName

\vspace{20mm}

\Large Bachelor's Thesis (9 ECTS)

\end{center}

\vspace{2mm}

\begin{flushright}
 {
 \setlength{\extrarowheight}{5pt}
 \begin{tabular}{r l} 
  \sffamily Supervisor: & \sffamily Vootele Rõtov, MSc \\
  \sffamily Supervisor: & \sffamily Vesal Vojdani, PhD
 \end{tabular}
 }
\end{flushright}

\vfill
\centerline{Tartu 2018}

% END TITLE PAGE

% Remember to remove this from the final thesis version
\pagebreak
\listoftodos[There's stuff to do!]
% END OF TODO PAGE 

% COMPULSORY INFO PAGE
\pagebreak

\selectlanguage{english}
\noindent\textbf{\large \articleName}
\vspace*{3ex}
\begin{flushleft}
  \textbf{Abstract:} 
\TODO{One or two sentences providing a basic introduction to the field, comprehensible to a scientist in
any discipline.}
\TODO{Two to three sentences of
more detailed background, comprehensible to scientists in related disciplines.}
\TODO{One sentence clearly stating the general problem being addressed by this particular
study.}
\TODO{One sentence summarising the main result (with the words here we show or their equivalent).}
\TODO{Two or three sentences explaining what
the main result reveals in direct
comparison to what was thought to be the case previously, or how the main result adds to previous knowledge.}
\TODO{One or two sentences to put the results into a more general context.}
\TODO{Two or three sentences to provide a
broader perspective, readily
comprehensible to a scientist in any
discipline, may be included in the first paragraph
if the editor considers that the accessibility of
the paper is significantly enhanced by their inclusion.}
\end{flushleft}


\vspace*{3ex}
\begin{flushleft}
  \textbf{Keywords:} \TODO{List of keywords}
\end{flushleft}
\vspace*{3ex}

\noindent\textbf{CERCS:} \TODO{CERCS kood ja nimetus:~\url{https://www.etis.ee/Portal/Classifiers/Details/d3717f7b-bec8-4cd9-8ea4-c89cd56ca46e}}
\selectlanguage{estonian}

\vspace*{5ex}
\noindent\textbf{\large \articleNameEE}
\vspace*{3ex}

\begin{flushleft}
  \textbf{Lühikokkuvõte:}
  \TODO{Abstract in estonian}
\end{flushleft}
\vspace*{3ex}

\begin{flushleft}
  \textbf{Võtmesõnad:} 
\end{flushleft}
\vspace*{3ex}

\noindent\textbf{CERCS:} 

\newpage

\selectlanguage{english}
\tableofcontents

\newpage
%Introduction
\section{Introduction}
\subfile{introduction/introduction.tex}

\pagebreak

\section{Sampling profiling methodologies}
\label{sec:profiling}
\subfile{profiling/profiling.tex}

\pagebreak


\section{Honest Profiler}
\label{sec:honest-profiler}
\subfile{honest-profiler/honest-profiler.tex}

\pagebreak
\section{Profiler optimization}
\label{sec:optimization}
\subfile{optimization/optimization.tex}

\pagebreak
\section{Future research}
\label{sec:future-research}
\subfile{future-research/future-research.tex}

\clearpage
\section{Conclusion} 

\subfile{conclusion/conclusion.tex}

\newpage

\bibliography{thesis}{}
\bibliographystyle{plain}

\newpage

\appendix
\section*{Appendices}
\addcontentsline{toc}{section}{Appendices}
% So that appendecies would be named by letters
\renewcommand{\thesubsection}{\Alph{subsection}}
\subfile{appendix/appendix.tex}
\pagebreak
\section*{\small Non-exclusive licence to reproduce thesis and make thesis public}

\TODO{License}

\end{document}
