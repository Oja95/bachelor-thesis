%!TEX root = ../thesis.tex
\documentclass[..thesis.tex]{subfiles}

%\subsection{\texttt{AsyncGetCallTrace} performance analysis}
Prior to increasing the sampling frequency, it is important to understand to what are the performance limitations that calling the \texttt{Async\-Get\-Call\-Trace} would imply. It is necessary to understand how long does a call to the \texttt{Async\-Get\-Call\-Trace} method take  to ensure that increasing the sampling frequency will not cause excessive performance overhead to the application being profiled. In order to measure the execution time, a custom Java Virtual Machine build with CPU time based time measurement logic was added to the \texttt{Async\-Get\-Call\-Trace} method. The added timing functionality is presented in Listing ~\ref{lst:asgct_time}.

\begin{lstlisting}[language=C++,style=def,label={lst:asgct_time}, caption={CPU time based measurement in \texttt{AsyncGetCallTrace} method}]
void AsyncGetCallTrace(ASGCT_CallTrace *trace, jint depth, void* ucontext) {
    struct timespec start, stop;
    clock_gettime(CLOCK_THREAD_CPUTIME_ID, &start);

    // AsyncGetCallTrace method code
    
    clock_gettime(CLOCK_THREAD_CPUTIME_ID, &stop);
    long accum = stop.tv_nsec - start.tv_nsec;
    printf("%li", accum);
}
\end{lstlisting}

Testing shows that a single \texttt{AsyncGetCallTrace} method call takes 4.6 microseconds on average.

%\subsection{Implications of the performance analysis}
Previously described limitations and \texttt{Async\-Get\-Call\-Trace} method performance measurements do not expose apparent problems or reasons on why \texttt{Async\-Get\-Call\-Trace} method can not be called more frequently during profiling. As the \texttt{Async\-Get\-Call\-Trace} method execution time is significantly smaller when compared to the default lowest achieveable sampling interval, the hypothesis is that that having smaller profiling interval will not affect the observed application's performance to a great extent. Thus, achieving higher sampling frequency is a goal worth pursuing.
