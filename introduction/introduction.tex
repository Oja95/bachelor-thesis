%!TEX root = ../thesis.tex
\documentclass[..thesis.tex]{subfiles}

\begin{document}

Profiling is an activity which aims to identify performance issues in an observable program. This task often relies on specific tools called profilers for extracting information from the program's execution. The output of the profiler helps to identify and locate the methods that have the largest effect on the program's execution time. Such methods are usually worth investigating as they affect the application's performance the most. \cite{mytkowicz_evaluating_2010}

As the results rely on the efficiency and accuracy of the tools used, it is important to have sufficient and actionable information from the profiler to construct accurate profiles of the application's performance. This thesis investigates a specific profiler implementation named Honest Profiler which is an open source sampling profiling tool which can be used to evaluate the performance of Java applications. The goal of this thesis is to increase the amount of information extracted from an observable Java application by Honest Profiler. Doing so increases the accuracy and reliability of the results. 

This thesis covers the basics of sampling profiling methods and the problems these methods have in the context of Honest Profiler. It explains the architecture of Honest Profiler and measures the performance of its profiling logic. The main result of this thesis is providing means to increase the amount of information that Honest Profiler can extract from the observable application. The suggested solutions are then tested on a benchmark test to evaluate their performance and amount of useful information extracted.

%\TODO{What is it in simple terms (title)?}
%\TODO{Why should anyone care?}
%\TODO{What was my contribution?} 
%\TODO{What you are doing in each section (a sentence or two per section)}
%\TODO{Consider creating a section for dumbed down version of what are we trying to achieve in here or cover in presenteatio n part}

\end{document}