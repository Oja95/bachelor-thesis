%!TEX root = ../thesis.tex
\documentclass[..thesis.tex]{subfiles}

\begin{document}

Honest Profiler is a software written in Java and C++ which aims to gather honest and accurate samples from a running Java application to provide an overview of the application's performance \cite{hon_prof}. Honest Profiler uses the \texttt{Async\-Get\-Call\-Trace} method for obtaining the samples. Moreover, Honest Profiler is licensed under an open source license which encourages the community to further investigate and develop this software solution. The fact that Honest Profiler is an open source software and utilizes a safepoint bias free method for obtaining the samples are the main reasons why this particular software is chosen for improvements and optimizations in this thesis.

\subsection{Architecture}
Honest Profiler consists of two larger components: a JVMTI native agent and a Java facade for result visualization, analysis and transformation. The profiling is almost completely done by the agent whereas the Java facade component aims to provide tools to ease the use of the profiler.

The JVMTI agent is a dynamic library which serves as a client of the JVM Tooling Interface \cite{jvmtm}. JVMTI provides these agents means to investigate and control the applications running on the Java Virtual Machine. Since the agents are written in native languages such as C and C++, they can also utilize various operating system's tools.

Honest Profiler relies on UNIX operating system signals and timers for obtaining samples.
It sets up a signal handler to handle \texttt{SIGPROF} signals sent to the Java process that the agent is attached to and an interval timer which sends the \texttt{SIGPROF} signal to that same process periodically. 

Upon receiving a signal, an \textit{arbitrary} thread currently executing on the CPU will handle the signal \cite{stevens_advanced_2013}. The signal handler proceeds to call \texttt{AsyncGetCallTrace} method from the JVMTI API to obtain the call trace of the current thread which is handling the signal. The obtained call trace is then persisted.

The thread handling the \texttt{SIGPROF} signal can be any thread of the Java process being profiled, even the threads responsible for garbage collection and just-in-time compilation. In such cases, the samples are discarded as such samples do not provide informative results. Samples which contain the observed application's stack frames are considered useful and provide relevant information about the application's performance.

%\supervisor{We currently hav a definition of useful sample in the future research section and we forward reference it from some point in next chapter. Maybe it would make sense to introduce this definitios somewhere here? Forward referencing to definition is not too nice and bringing in ``new'' definitions in future research does not seem ideal.}

\subsection{Current limitations}
Current implementation of Honest Profiler makes use of the standard CPU time based interval timer from GNU time library and UNIX operating system signals to periodically call the \texttt{AsyncGetCallTrace} method. CPU time based timer counts down time only when the process is consuming CPU time which means that the timer's value will only be incremented when the process is executing its instructions on the processor.

The lowest possible interval for the CPU based interval timer is the time between two ticks of the system's internal timer interrupt (\textit{jiffy}). This also implies that the lowest possible \texttt{Async\-Get\-Call\-Trace} method call interval is also the duration of a jiffy. This duration depends on the interrupt frequency of the hardware platform being used. Default interrupt frequency for Linux kernel based operating systems (since kernel version 2.6.13) is 250 Hz which results in $\frac{1}{250} = 0.004$ seconds for the duration of a jiffy \cite{linux_time}. In such configuration, minimum sampling interval would be 4 milliseconds.

Therefore, increasing the sampling frequency will require significant changes in the architecture of how timer signals are sent to the profiler or in the kernel's confugration. The following sections will describe that different implementations not only perform differently in terms of overhead but also in terms of which thread ends up receiving timer signals. Hence, the ratio of usable stack traces versus those that have to be discarded are different for each implementation.

\end{document}
